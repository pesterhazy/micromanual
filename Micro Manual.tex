\documentclass[8pt,letter,twocolumn]{article}
\usepackage[margin=0.6in]{geometry}
\usepackage[utf8]{inputenc}
\usepackage{fancyvrb}
\usepackage{changes} % For \textsubscript

\usepackage[protrusion=true,expansion=true,activate={true,nocompatibility},final,factor=1200,stretch=35,shrink=35]{microtype}
\doublehyphendemerits=10000

\title{\small \textbf{\textsc{{A Micro-Manual for LISP - Not the whole truth}}}}

\author{\footnotesize \bfseries John McCarthy\\
            \footnotesize \bfseries Artificial Intelligence Laboratory\\
            \footnotesize \bfseries Stanford University}

\date{}

\newcommand{\args}[1] {#1\textsubscript{i} \dots \ #1\textsubscript{n}}
\newcommand{\ei}[0] {e\textsubscript{i}}
\newcommand{\en}[0] {e\textsubscript{n}}

\begin{document}
\setcounter{page}{215}
\maketitle

% \section{Introduction}

LISP data are symbolic expressions that can be either \textit{atoms} or
\textit{lists}. \textit{Atoms} are strings of letters and digits and other
characters not otherwise used in LISP. A list consists of a left parenthesis
followed by zero or more atoms or lists separated by spaces and ending with a
right parenthesis. Examples: \texttt{a}, \texttt{onion}, \texttt{()},
\texttt{(a)}, \texttt{(a onion a)}, \texttt{(plus 3 (times x pi) 1)},
\texttt{(car (quote (a b)))}.

The LISP programming language is defined by rules whereby certain LISP
expressions have other LISP expressions as \textit{values}. The function called
\textit{value} that we will use in giving these rules is not part of the LISP
language but rather part of the informal mathematical language used to define
LISP. Likewise, the italic letters \textit{e} and \textit{a} (sometimes with
subscripts) denote LISP expressions, the letter \textit{v} (usually subscripted)
denotes an atom serving as a variable, and the letter \textit{f} stands for a
LISP expression serving as a function name.

\begin{enumerate}
\setlength\itemsep{0em}
\item \textit{value} \texttt{(quote \textit{e})} = \textit{e}. Thus the value of
  \texttt{(quote a)} is \texttt{a}.

\item \textit{value} \texttt{(car \textit{e})}, where \textit{value e} is a
  non-empty list, is the first element of \textit{value e}. Thus \textit{value}
  \texttt{(car (quote (a b c)))} = \texttt{a}.

\item \textit{value} \texttt{(cdr e)}, where \textit{value} \textit{e} is a
  non-empty list, is the the list that remains when the first element of
  \textit{value} \textit{e} is deleted. Thus \textit{value} \texttt{(cdr (quote
    (a b c)))} = \texttt{(b c)}.

\item \textit{value} \texttt{(cons el e2)}, is the list that results from
  prefixing \textit{value e1} onto the list \textit{value e2}. Thus
  \textit{value} \texttt{(cons (quote a) (quote (b c)))} = \texttt{(a b c)}.

\item \textit{value} \texttt{(equal el e2)} is T if \textit{value e1} =
  \textit{value e2}. Otherwise, its \textit{value} is \texttt{NIL}. Thus
  \textit{value} \texttt{(equal (car (quote (a b))) (quote a))} = \texttt{T}.

\item \textit{value} \texttt{(atom e)} = \texttt{T} if \textit{value e} is an
  atom; otherwise its \textit{value} is \texttt{NIL}.

\item \textit{value} \texttt{(COND(p\textsubscript{i} \ei) \dots
    (p\textsubscript{n} \en))} = \textit{value} \texttt{\ei}, where
  \texttt{p\textsubscript{i}} is the the first of the \textit{p}\rq s whose
  \textit{value} is not \texttt{NIL}. Thus \textit{value} \texttt{(cond ((atom
    (quote a)) (quote b)) ((quote t) (quote c)))} = \texttt{b}.

\item An atom \textit{v}, regarded as a variable, may have a value.

\item \textit{value} \texttt{((lambda (\args{v}) e) \args{e})} is the same as
  \textit{value e} but in an environment in which the variables \args{v} take
  the values of the expressions \args{e} in the original environment. Thus
  \textit{value} \texttt{((lambda (x y) (cons (car x) y)) (quote (a b)) (cdr
    (quote (c d))))} = \texttt{(a d)}.

\item Here\rq s the hard one. \textit{value} \texttt{((label \textit{f} (lambda
    (\args{v}) e)) \args{e})} is the same as \textit{value} \texttt{((lambda
    (\args{v}) e) \args{e})} with the additional rule that whenever
  \texttt{(\textit{f} \args{a})} must be evaluated, \textit{f} is replaced by
  \texttt{(LABEL \textit{f} (lambda (\args{v}) e))}. Lists beginning with
  \texttt{label} define functions recursively.

\end{enumerate}

This is the core of LISP, and here are more examples:  \\

\textit{value} \texttt{(car x)} = \texttt{(a b)} if \textit{value} \texttt{x} =
\texttt{((a b) c)}, and \textit{value} \texttt{((label ff (lambda (x) (cond
  ((atom x) x) ((quote t) (ff (car x)))))) (quote ((a b) c)))}= \texttt{a}

Thus \texttt{((label ff (lambda (x) (cond ((atom x) x) ((quote t) (ff (car
  x))))))}, is the LISP name of a function \textit{ff} such that \textit{ff e}
is the first atom in the written form of \textit{e}. Note that the list
\textit{ff} is substituted for the atom \texttt{ff} twice.

Difficult mathematical type exercise: Find a list \textit{e} such that
\textit{value} \textit{e} = \textit{e}.

\vspace*{1\baselineskip}
\textbf{Abbreviations}
\vspace*{1\baselineskip}

The above LISP needs some abbreviations for practical use.

\begin{enumerate}
\setlength\itemsep{0em}

\item The variables \texttt{T} and \texttt{NIL} are permanently assigned the
  values \texttt{T} and \texttt{NIL}, and \texttt{NIL} is the name of the null
  list \texttt{()}.

\item So as not to describe a LISP function each time it is used, we define it
  permanently by typing \texttt{(defun f (\args{v}))) e)} . Thereafter
  \texttt{(f \args{e})} is evaluated by evaluating \textit{e} with the variables
  \args{v} taking the values value \textit{value} \ei, \dots, \textit{value} \en
  respectively. Thus, after we define \texttt{(defun ff (x) (cond ((atom x) x)
    (t (ff (car x)))))}, typing \texttt{(ff (quote (ca b) c)))}, gets \texttt{a}
  from LISP.

\item We have the permanent function definitions.

  \texttt{(defun null (x) (equal x nil))} and \texttt{(defun cadr (x) (car (cdr
    x)))}.

  and similarly for arbitrary combinations of A and D.

\item \texttt{(list \args{e})} is defined for each n to be \texttt{(cons \ei
    (cons \dots \ (cons \en nil)))} .

\item \texttt{(and p q)} abbreviates \texttt{(cond (p q) (t nil))}.
  \textit{AND}s with more terms are defined similarly, and the propositional
  connectives \texttt{or} and \texttt{not} are used in abbreviating
  corresponding conditional expressions.

\end{enumerate}

Here are more examples of LISP function definitions:

\texttt{(defun alt (x) (cond ((or (null x) (null (cdr x))) x) (t (cons (car x)
  (alt (cddr x))))))}

defines a function that gives alternate elements of a list starting with the
first element. Thus \texttt{(alt (quote (a b c d e)))} = \texttt{(a c e)}

\texttt{(defun subst (x y z) (cond ((atom z) (cond ((equal z y) x) (t z))) (t
  (cons (subst x y (car z))(subst x y (cdr z))))))},

where Y is an atom, gives the result of substituting X for Y in Z. Thus
\texttt{(subst (quote (plus x y)) (quote v) (quote (times x v)))} =
\texttt{(times x (plus x y))}.

You may now program in LISP. Call LISP on a time-sharing computer, define some
functions, type in a LISP expression, and LISP will output its value on your
terminal.

\vspace*{1\baselineskip}
\textbf{The LISP interpreter written in LISP}
\vspace*{1\baselineskip}

The rules we have given for evaluating LISP expressions can themselves be
expressed as a LISP function \texttt{(eval e a)}, where \textit{e} is an
expression to be evaluated, and \textit{a} is a list of variable-value pairs,
\textit{a} is used in the recursion and is often initially \texttt{NIL}. The
long LISP expression that follows is just such an evaluator. It is presented as
a single LABEL expressions with all auxiliary functions also defined by LABEL
expressions internally, so that it references only the basic function of LISP
and some of abbreviations like CADR and friends. It knows about all the
functions that are used in its own definition so that it can evaluate itself
evaluating some other expression. It does not know about DEFUNs or any features
of LISP not explained in this micro-manual such as functional arguments,
property list functions, input-output, or sequential programs.

The function \texttt{eval} can serve as an interpreter for LISP, and LISP
interpreters are actually made by hand-compiling EVAL into machine language or
by cross-compiling it on a machine for which a LISP system already exists.

The definition would have been easier to follow had we defined auxiliary
functions separately rather than include them using LABEL. However, we would
then have needed property list functions in order to make the \texttt{eval}
self-applicable. These auxiliary functions are \texttt{evlis} which evaluates
lists of expressions, \texttt{evcond} which evaluates conditional expressions,
\texttt{assoc} which finds the value associated with a variable in the
environment, and \texttt{pairup} which pairs up the corresponding elements of
two lists.

\newpage
Here is \texttt{eval}.

\VerbatimInput[fontfamily=tt, fontsize=\scriptsize, samepage=true]{eval.lisp}

\end{document}
